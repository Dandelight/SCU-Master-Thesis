% 文件名:MainBody.tex
% 文件描述:四川大学2020研究生硕/博士 LaTeX 模版
% 作者:Yu Wang
% 修改日期:2023年
 
% 设置文档属性
% 参数说明
% professional: 专业学位
% academic: 学术学位
% master: 硕士
% doctor: 博士
% approval: 送审版本,将不生成声明
% secret: 保密论文,将显示密级
% color: 红色川大logo
\documentclass[academic,doctor,color]{./Template/scuthesis2020}


 % AMS Package
\usepackage{amsfonts}
\usepackage{amsmath,amsfonts,amssymb,amscd}
\usepackage{latexsym,mathrsfs, amscd,amsthm,verbatim}

\usepackage{ragged2e}

%调整间距
%\topmargin 2 cm
\setlength{\parskip}    {3pt plus1pt minus2pt}  %段落间距
%\setlength{\baselineskip}    {20pt plus2pt minus1pt} %字号的基本距离
%\renewcommand{\baselineskip}{2} %字号的基本距离
%\setlength{\textheight} {21.5true cm}               %文本区域高度21.5cm
%\setlength{\textwidth}  {14.4true cm}               %文本区域宽度14.5cm

% 解决fancy抱怨说标题高度太低的问题
\setlength{\headheight}{15pt}

\allowdisplaybreaks

% 一些数学记号
\def\deq{\mathop{\buildrel\Delta\over=}}

\begin{document}
% 设置文档信息
\unitid{10610} % 单位代码
\STUnumber{STUnumber} % 学号或送审编号
\securityClassification{秘密} % 密级:公开/内部/秘密/机密/绝密. 当不使用secret时,不显示
\securityYear{3} % 保密年限
\CoverTitle{CoverTitle} %封面标题
\title{} % 论文全称
\ENGtitle{ENGtitle} %论文全称英文
% \school{\zihao{5}{这里是特别长的单位名称甚至一行不能显示}} % 如果太长超出一行,可以根据显示效果修改字号,这里为5号字
\school{school} % 培养单位
\ENGschool{Computer Science} % 培养单位英文
\author{author} % 作者姓名
\ENGauthor{ENGauthor} % 作者英文名
\supervisor{supervisor} % 指导教师
\ENGsupervisor{ENGsupervisor} % 指导教师英文
\degreeclass{degreeclass} % 学位类别
\ENGdegreeclass{Master of Engineering} % 学位类别英文
\major{major} % 学科专业或领域名称
\ENGmajor{ENGmajor} % 学科专业或领域名称英文
\hasmajor{1} % 若有领域则为1,否则改为0
\researchDirection{researchDirection} % 若有领域则为1,否则改为0
\completedate{2023年5月6日}
\defensedate{二〇二三年五月} % 论文答辩时间
\grantdate{二〇二三年五月} % 学位授予时间
\accomplishdate{二〇二三年三月} % 论文完成时间
\statementdate{February, 2020} % 声明时间
\direction{XXX}
\ENGdirection{Direction Name}
\keywords{keywords }
\ENGkeywords{ENGkeywords}

% 自动制作封面
\maketitle

% 设置论文正文前的页码、页眉等
\frontmatter\pagenumbering{Roman}\pagestyle{fancy}

% todo 列表页,如果要显示,需要在文档类中加入todo选项
\showToDoListPage

% 包含摘要
%!TEX root = ../MainBody.tex

% 中英文摘要
\begin{CHSabstract}

    
    CHSabstract
    

\end{CHSabstract}

\begin{ENGabstract}	
    
	ENGabstract

\end{ENGabstract}


% 自动制作目录
\maketoc
% 自动制作图表目录
%\makefigtablist
% 包含缩略词表
%\include{Chapters/0_1_Abbreviations}
% 包含符号表
%\include{Chapters/0_2_Symbols}

% 设置论文正文部分的页码、页眉等
\mainmatter\pagenumbering{arabic}\pagestyle{fancy}

% 包含第一章、第二章等等
% 第一章    
\chapter{绪论}

\section{研究背景与研究概况}


\section{主要研究工作}

本节将介绍本文的主要研究工作.


\cite{Barbu2003}

\chapter{总结与展望}

在本章中, 我们总结了本文的研究成果, 并提出了一些未来的研究方向.

\section{总结}


\appendix

\chapter{技术性结论的证明}

本章节中, 我们给出前面章节中的一些技术性结论的证明.



% 设置论文正文后的式样
\backmatter
% 按国标自动制作参考文献
% 参考文献数据文件为本目录下的ReferenceBase.bib
\begin{reference}
	\bibliography{reference}
\end{reference}


% 包含在读期间科研成果
%!TEX root = ../MainBody.tex

% 作者在读期间科研成果简介
\chapter{攻读学位期间取得的研究成果}

\begin{itemize}
	\item[1.] 攻读学位期间取得的研究成果

	
	
\end{itemize}


% 包含致谢
\makethanks

% 自动制作中英文声明
\makestatement

\end{document}
